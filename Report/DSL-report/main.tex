%%%%%%%%%%%%%%%%%%%%%%%%%%%%%%%%%%%%%%%%%%%%%%%%%%%%%%%%%%%%%%%%%%%%%%%
% Based on IEEE the conference template available                     %
% at https://www.ieee.org/conferences/publishing/templates.html       %
% Adapted for the Data Science Lab course at Politecnico di Torino    %
% by Giuseppe Attanasio, Flavio Giobergia                             %
% 2020, DataBase and Data Mining Group                                %
%%%%%%%%%%%%%%%%%%%%%%%%%%%%%%%%%%%%%%%%%%%%%%%%%%%%%%%%%%%%%%%%%%%%%%%

\documentclass[conference]{IEEEtran}
\usepackage{cite}
\usepackage{amsmath,amssymb,amsfonts}
\usepackage{algorithmic}
\usepackage{graphicx}
\usepackage{textcomp}
\usepackage{xcolor}

\begin{document}

\title{Sentiment prediction of Twitter contents}

\author{\IEEEauthorblockN{Edoardo Chiò}
\IEEEauthorblockA{\textit{Politecnico di Torino} \\
Student id: s301486 \\
s301486@studenti.polito.it}
}

\maketitle

\begin{abstract}
The abstract goes here. Keep it short (approx. 3-4 sentences)
\end{abstract}

\section{Problem overview}
Here you should describe your problem

\section{Proposed approach}
In this section, you will present your solution. Please fill in accordingly.

You can use citations as follows: \cite{goodfellow2016deep} (you can add BibTeX citations in the \textit{bibliography.bib} file).

\subsection{Preprocessing}
\subsection{Model selection}
\subsection{Hyperparameters tuning}

\section{Results}
Here you will present your results (models \& configurations selected, performance achieved)

\section{Discussion}
Any relevant discussion goes here.

\bibliography{bibliography}
\bibliographystyle{ieeetr}

\end{document}
